
\documentclass{tufte-handout}
\usepackage{amsmath,amsthm}

\usepackage{booktabs}
\usepackage{graphicx}
\usepackage{tikz}

\newtheorem{claim}{Claim}[section]
\title{\sf Marking Trees}
\date{}
\begin{document}
\maketitle

\section{Lab Report: Marking Trees}


by Stefan Eng, {name}, {name} and {name}.

\subsection{Results}

For $i\in\{1,2,3\}$, the number of rounds $R_i$ spent until the tree
is completely marked in process $i$ is given in the following table.
The table shows the result of 24 repeated
trails.

In the last column, report the expected value of $R_1$ for each $N$, using the
formula derived from your theoretical analysis in the following section

\medskip\noindent
\begin{tabular}{ l l l l l l }
% WARNING: This table the (brilliant) siunitx package.
% This allows typesetting of nicely aligned numbers.
% If this is too much to absorb, just use a normal Latex table.
% (Or do the table in another tool, export as PDF, and include it.)
% Or do the whole report in your favourite word processor instead.
\toprule
 { $N$ } & { $R_1$ } & {$R_2$} & {$R_3$} & {$\mathbf E[R_1]$}\\\midrule
3 & 2.40e+00 $\pm$ 8.00e-01 & 2.00e+00 & 2.00e+00 & 0.00e+00 & \\
7 & 5.60e+00 $\pm$ 1.85e+00 & 5.00e+00 $\pm$ 8.94e-01 & 4.00e+00 & 3.50e+00 & \\
15 & 2.48e+01 $\pm$ 1.00e+01 & 9.80e+00 $\pm$ 1.17e+00 & 8.00e+00 & 1.38e+01 & \\
31 & 3.90e+01 $\pm$ 9.94e+00 & 2.06e+01 $\pm$ 1.36e+00 & 1.60e+01 & 4.02e+01 & \\
63 & 1.01e+02 $\pm$ 2.01e+01 & 4.82e+01 $\pm$ 7.47e+00 & 3.20e+01 & 1.05e+02 & \\
127 & 2.57e+02 $\pm$ 2.69e+01 & 1.08e+02 $\pm$ 4.82e+00 & 6.40e+01 & 2.56e+02 & \\
255 & 6.92e+02 $\pm$ 2.43e+02 & 2.20e+02 $\pm$ 1.08e+01 & 1.28e+02 & 6.03e+02 & \\
511 & 1.52e+03 $\pm$ 2.37e+02 & 4.69e+02 $\pm$ 5.61e+00 & 2.56e+02 & 1.39e+03 & \\
1023 & 3.23e+03 $\pm$ 4.80e+02 & 9.63e+02 $\pm$ 4.91e+01 & 5.12e+02 & 3.13e+03 & \\
2047 & 7.36e+03 $\pm$ 1.68e+03 & 1.96e+03 $\pm$ 5.15e+01 & 1.02e+03 & 6.97e+03 & \\
4095 & 1.82e+04 $\pm$ 3.27e+03 & 3.98e+03 $\pm$ 3.25e+01 & 2.05e+03 & 1.54e+04 & \\
8191 & 3.73e+04 $\pm$ 3.78e+03 & 8.00e+03 $\pm$ 1.01e+02 & 4.10e+03 & 3.36e+04 & \\
16383 & 7.58e+04 $\pm$ 4.11e+03 & 1.61e+04 $\pm$ 1.98e+02 & 8.19e+03 & 7.29e+04 & \\
32767 & 1.41e+05 $\pm$ 1.31e+04 & 3.24e+04 $\pm$ 2.19e+02 & 1.64e+04 & 1.57e+05 & \\
65535 & 3.23e+05 $\pm$ 2.61e+04 & 6.51e+04 $\pm$ 1.66e+02 & 3.28e+04 & 3.37e+05 & \\
131071 & 7.38e+05 $\pm$ 8.30e+04 & 1.31e+05 $\pm$ 2.71e+02 & 6.55e+04 & 7.19e+05 & \\
262143 & 1.39e+06 $\pm$ 1.12e+05 & 2.61e+05 $\pm$ 4.98e+02 & 1.31e+05 & 1.53e+06 & \\
524287 & 2.93e+06 $\pm$ 1.59e+05 & 5.23e+05 $\pm$ 4.30e+02 & 2.62e+05 & 3.24e+06 & \\
1048575 & 6.74e+06 $\pm$ 5.83e+05 & 1.05e+06 $\pm$ 4.51e+02 & 5.24e+05 & 6.84e+06 & \\

\bottomrule
\end{tabular}

\subsection{Analysis}

\includegraphics[width=\textwidth]{values_R1.pdf}
\includegraphics[width=\textwidth]{values_R2.pdf}
\includegraphics[width=\textwidth]{values_R3.pdf}

Our experimental data indicates that $\mathbf E [R_1] = 2\frac{1}{4}\cdot
nH_n = \frac{n}{2}H_n$ \\
$\mathbf E[R_2]$ is bounded by: $ \frac{n}{2} \le \mathbf E[R_2] \le n ,$
and $\mathbf E[R_3] = \frac{n}{2}$.
\\
\
\\

Theoretically, the behaviour of $R_1$ can be explained as follows:
$\cdots$

\end{document}
